\chapter{Results and Evaluation}

\begin{itemize}
  \item different datasets: checkersphere, virtual room, real data
  \item use evaluation in order to determine the effect of the different parameters on quality? \ar relative quality
  \item compare flow-based and regular 2D interpolation
  \item can only compare if flow algorithm works more or less correctly \ar narrows the parameter space for comparison
\end{itemize}

Parameters that (may) have an effect on quality
\begin{itemize}
  \item distance between captures / ``resolution'' of captures
  \item distance of scene from model (checkersphere, square room, arbitrary room) and accuracy of closest radius (were captures actually taken near edges of the scene?)
  \item weight function
  \item accuracy of capture locations
  \item actual distance from scene to viewpoint (are there objects close to the camera)
  \item accuracy of optical flow algorithm
  \item number of samples/viewpoints used for interpolation (closest, within radius, etc)
\end{itemize}

Process:
\begin{itemize}
  \item first, for each scene, determine resolution for which optical flow still works by using 1D interpolation 
  \item diagonal movement, left to right, front to back for each resolution setting
  \item interpolation in 0.1 steps \ar render out (or record) correct values
  \item where in the scene? close to the walls? in the middle?
\end{itemize}<++>

\section{Optical Flow}
\begin{itemize}
  \item optical flow ground truth is impossible to get from real scenes
  \item however, virtual scenes contain all necessary information for retrieving ground truth for optical flow
  \item virtual camera rig that captures one image per ``side'' with a fov that corresponds to that used in the ExtendedCubeMap \ar extended flow cube
\end{itemize}

"correctness" of optical flow interpolation is limited even with perfect optical flow:
\begin{itemize}
   \item points that are not visible because of perspective shift will not have a correspondence
   \item the actual trajectory of a specific point is not necessarily linear, but due to the nature of the algorithm, the movement will always be linear
\end{itemize}
errors that are unrelated to the algorithm:
\begin{itemize}
  \item slight displacement due to ExtendedCubeMap
  \item black edges due to latlong-cube conversion
  \item these need to be taken into account (normalized out)
\end{itemize}
