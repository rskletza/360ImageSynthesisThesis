\chapter{Evaluation and Discussion}

\section{Evaluation}
\subsection{Parameters}
Parameters used in evaluation
\begin{itemize}
  \item viewpoint density / ``resolution'' of captures
  \item distance of scene from model (checkersphere, square room, arbitrary room)
  \item number of samples/viewpoints used for interpolation \ar two vs all within semi-large radius (all would be very computation intensive)
  \item flow-based blending vs deviation-angle-based blending
  \item only makes sense to compare if flow algorithm works more or less correctly \ar narrows the parameter space for comparison
\end{itemize}

Fixed params:
\begin{itemize}
  \item radius accuracy
  \item weight metric only deviation angle, not distance (although distance is limited by viewpoint choice)
  \item only 2D
  \item metadata is present and correct
\end{itemize}

\subsection{Error Calculation}

\begin{verbatim}
- no publicly available benchmarks for image-based rendering for 360\degree images
- create metrics to measure quality
- create datasets to test algorithm with different parameters
- these datasets could be used in the future to compare different algorithms
\end{verbatim}

\subsection{Synthesizing Optical Flow}
\begin{itemize}
  \item optical flow ground truth is impossible to get from real scenes
  \item however, virtual scenes contain all necessary information for retrieving ground truth for optical flow
  \item virtual camera rig that captures one image per ``side'' with a fov that corresponds to that used in the ExtendedCubeMap \ar extended flow cube
\end{itemize}

``correctness'' of optical flow interpolation is limited even with ``perfect'' optical flow:
\begin{itemize}
   \item points that are not visible because of perspective shift will not have a correspondence
   \item distortion due to wide fov may have an effect on the results
   \item even blender motion vectors are for frame to frame use, so it is possible, that large jumps do not work well because the blender algo can't handle it
\end{itemize}

\subsection{Scenarios}
\subsubsection{Minimum vs Maximum Number of Input Viewpoints}
\begin{verbatim}
Hypothesis: 2vps is better for flow-based, radius is better for regular. flow based does not do well with image patches with different viewpoint indices (extreme discontinuities)

Test: 
  -use the two simplest scenes with a medium number of viewpoints and synthesize using only two vs all within the radius of 1/2*scene radius

Show: 
  - compare flow to flow and regular to regular --> we only want to know which is better for each blending method, not in comparison
  - y axis: box plot for error
  - x axis: checkersphereRviewpoints, checkersphere2viewpoints, squareroomRviewoints, squareroom2viewpoints, maybe more

Even if the hypothesis is not proven, or is not very distinct, we can still argue that >2 or >4 is computationally too costly for consideration

\end{verbatim}

\subsubsection{Viewpoint density effect on flow-based blending and regular blending}
\begin{verbatim}
Hypothesis: higher density --> better results, but difference is larger for flow-based. (optional: High density and 2vps may be better than 4)
Hypothesis II: the effect of higher density is more marked near walls and corners

Test:
  - use square room and test 2x2, 6x6, 10x10
  - compare flow with flow, reg with flow, and reg with reg -> 3 graphs

\end{verbatim}<++>

\subsubsection{Flow-based blending vs regular depending on model-scene difference}
\begin{verbatim}
Hypothesis: the closer the scene is to the model, the better regular blending is. the farther away, the worse it is and the better flow-based blending is

Test: compare the two with their respective better vp selection in the different scenes

\end{verbatim}


\section{Discussion}

\subsection{Limits}

limits of the evaluation:
\begin{itemize}
  \item objects are not clearly classified/measured, so no quantifiable evaluation possible
  \item positional and rotational knowledge may be calculable by sfm algos, but we don't know what kind of impact accuracy of the metadata will have, since we are hand-recording metadata meticulously
  \item pixel differences and ssim give no indication on human perception, so it is not possible to judge believability
\end{itemize}

Assuming radius accuracy does in fact make a slight assumption about the scene geometry. Using only the deviation angle will lead to ``spots'' where the distance is fairly large but the angle is zero. 

\subsection{Future Work}
\begin{itemize}
  \item ``guess'' an optimal radius without using viewpoint locations e.g. outside
  \item find a good weight function that balances deviation angle and distance appropriately
  \item figure out a good way to find a ``left and right'' for 3D, also make it more stable for 2D (so that viewpoints are chosen that are relatively close to one another so that the optical flow algorithm works better)
  \item use methods like SLAM in order to be independent of actually recording metadata by hand
\end{itemize}

