\chapter{Synthesized Images}\label{imgs}

\begin{figure}
		\centering
		\includegraphics[width=\textwidth]{04/inspection_example_K.jpg}
		\caption{Sample inspection of example viewpoint ``K'': The images are in cube map representation, as this is tends to be more intuitive to understand than latlong representation. The top face is omitted for a more compact representation.}
		\label{fig:inspection_example}
\end{figure}

\begin{figure}
		\centering
    \includegraphics[width=\textwidth]{04/scenario_scene/stripX_checkersphere_Y.jpg}
		\caption{Results for synthesized viewpoint ``Y'' in the checkersphere scene: The regular blending result is very close to the ground truth, except for some blurriness. The flow-based blending result shows some inaccuracies (magenta) and noise (green)}
		\label{fig:scene_checkersphere_Y}
\end{figure}

\begin{figure}
		\centering
    \includegraphics[width=\textwidth]{04/scenario_scene/O_regular_both_scenes.jpg}
		\caption{Regular blending result of viewpoint ``O'' in the square and oblong rooms: The bookshelf has a strong impact on the difference in error values (marked in green)}
		\label{fig:scene_O_regular}
\end{figure}

\begin{figure}
		\centering
    \includegraphics[width=\textwidth]{04/scenario_scene/O_flow_both_scenes.jpg}
		\caption{Flow-based blending result of viewpoint ``O'' in the square and oblong rooms: The bookshelf has a strong impact on the difference in error values (green) and the details of the bookshelf are warped due to inaccurate optical flow (magenta).}
		\label{fig:scene_O_flow}
\end{figure}

%\begin{figure}
%\centering
%    \hfill
%    \begin{subfigure}[b]{\textwidth}
%            \centering
%            \includegraphics[width=0.9\textwidth]{04/scenario_scene/stripX_square_synth_room_O.jpg}
%            \caption{Square room: The proximity to the bookshelf results in extreme inaccuracies in the synthesis}
%    \end{subfigure}
%    \hfill
%
%    \hfill
%    \begin{subfigure}[b]{\textwidth}
%            \centering
%            \includegraphics[width=0.9\textwidth]{04/scenario_scene/stripX_oblong_room_O.jpg}
%            \caption{Oblong room: The bookshelf is farther away from the viewpoint, which improves the accuracy immensely}
%    \end{subfigure}
%    \hfill
%  \caption{Comparing the viewpoints at location ``O'' in the square and oblong rooms} \label{fig:scene_square_oblong_O}
%\end{figure}

\begin{figure}
  \centering
  \includegraphics[width=0.9\textwidth]{04/scenario_scene/stripX_square_synth_room_N.jpg}
  \caption[Synthesized point ``N'' in the square room]{Synthesized point ``N'' in the square room (best improvement for L1, good for SSIM): The flow-based blending removed the ghosting artefacts on the rug (magenta) and improved the accuracy on the coffee table (cyan), and the lower part of the bookshelf (green). The rest of the scene is very similar for both results.}
  \label{fig:scene_square_N}
\end{figure}

\begin{figure}
  \centering
  \includegraphics[width=0.9\textwidth]{04/scenario_scene/stripX_square_synth_room_L.jpg}
  \caption[Synthesized point ``L'' in the square scene]{Synthesized point ``L'' in the square room (worst ``improvement'': slight increase of error for both metrics): The ghosting artefact on the rug in the regular blending result was replaced by a displacement artefact in the flow-based blending (magenta), which also introduced a new artefact, namely the warped top edge of the coffee table (green). Otherwise the scenes are very similar.}
  \label{fig:scene_square_L}
\end{figure}

\begin{figure}
  \centering
  \includegraphics[width=0.9\textwidth]{04/scenario_scene/stripX_oblong_room_A.jpg}
  \caption{Synthesized point ``A'' in the oblong room (best improvement): The flow-based blending drastically improved ghosting artefacts on the couch (magenta) and the accuracy of the rug (cyan), but also introduced new artefacts (green).}
  \label{fig:scene_oblong_A}
\end{figure}

\begin{figure}
  \centering
  \includegraphics[width=0.9\textwidth]{04/scenario_scene/stripX_oblong_room_L.jpg}
  \caption{Synthesized point ``L'' in the oblong room (worst improvement): The flow-based blending result introduced some severe discontinuity artefacts on the rug (magenta) and on the coffee table (cyan), although the coffe table is positioned more accurately in the flow-based blending result (green)}
  \label{fig:scene_oblong_L}
\end{figure}

\begin{figure}
		\centering
    \includegraphics[width=\textwidth]{04/scenario_density/strip_dens_T_regular.jpg}
		\caption{The regular blending results for point ``T'' (one of the worse results) in the square room with a viewpoint density of 2x2, 6x6, and 12x12}
		\label{fig:density_regular_T}
\end{figure}

\begin{figure}
		\centering
    \includegraphics[width=\textwidth]{04/scenario_density/strip_dens_G_regular.jpg}
		\caption{The regular blending results for point ``G'' (one of the better results) in the square room with a viewpoint density of 2x2, 6x6, and 12x12}
		\label{fig:density_regular_G}
\end{figure}

\begin{figure}
\centering
    \hfill
    \begin{subfigure}[b]{\textwidth}
            \centering
            \includegraphics[width=0.9\textwidth]{04/scenario_density/stripX_2x2v2_offset_new_T.jpg}
            \caption{Synthesized point ``T'' (best improvement for L1, good for SSIM)}
    \end{subfigure}
    \hfill

    \hfill
    \begin{subfigure}[b]{\textwidth}
            \centering
            \includegraphics[width=0.9\textwidth]{04/scenario_density/stripX_2x2v2_offset_new_K.jpg}
            \caption{Synthesized point ``K'' (worst ``improvement'': slight increase of error)}
    \end{subfigure}
    \hfill
  \caption[Best and worst improvements of flow-based blending over regular blending in the 2x2 setup]{Best (T) and worst (K) improvements of flow-based blending over regular blending with the 2x2 setup in the square room. The ``worst'' improvement is slightly worse than the regular blending result} \label{fig:density_2x2_best_worst}
\end{figure}

\begin{figure}
\centering
    \hfill
    \begin{subfigure}[b]{\textwidth}
            \centering
            \includegraphics[width=0.9\textwidth]{04/scenario_density/stripX_12x12_offset_new_Y.jpg}
            \caption{Synthesized point ``Y'' (best improvement for L1 and SSIM)}
    \end{subfigure}
    \hfill

    \hfill
    \begin{subfigure}[b]{\textwidth}
            \centering
            \includegraphics[width=0.9\textwidth]{04/scenario_density/stripX_12x12_offset_new_H.jpg}
            \caption{Synthesized point ``H'' (worst ``improvement'': slight increase of error)}
    \end{subfigure}
    \hfill
  \caption[Best and worst improvements of flow-based blending over regular blending with the 12x12 setup]{Best (Y) and worst (H) improvements of flow-based blending over regular blending in the 12x12 setup in the square room. The ``worst'' improvement is slightly worse than the regular blending result} \label{fig:density_12x12_best_worst}
\end{figure}

\begin{figure}
\centering
    \hfill
    \begin{subfigure}[b]{\textwidth}
            \centering
            \includegraphics[width=0.9\textwidth]{04/scenario_offset/stripX_6x6_dense_52_good.jpg}
            \caption{The coffee table is in a more accurate position, even though it shows some blurriness}
    \end{subfigure}
    \hfill

    \hfill
    \begin{subfigure}[b]{\textwidth}
            \centering
            \includegraphics[width=0.9\textwidth]{04/scenario_offset/stripX_6x6_dense_567_good.jpg}
            \caption{The rug no longer has doubled edges, however, there are still some artefacts}
    \end{subfigure}
    \hfill
  \caption{Results for which the flow-based blending improved the accuracy} \label{fig:offset_good}
\end{figure}

\begin{figure}
\centering
    \hfill
    \begin{subfigure}[b]{\textwidth}
            \centering
            \includegraphics[width=0.9\textwidth]{04/scenario_offset/stripX_6x6_dense_145_bad.jpg}
            \caption{The results are very similar, except on the blue table in the bottom face, and the white coffee table in the left face, where there are distinct artefacts in the flow-based result.}
    \end{subfigure}
    \hfill

    \hfill
    \begin{subfigure}[b]{\textwidth}
            \centering
            \includegraphics[width=0.9\textwidth]{04/scenario_offset/stripX_6x6_dense_504_bad_ok.jpg}
            \caption{In this case, there is hardly a visible difference between the two results}
    \end{subfigure}
    \hfill
  \caption{Results for which the flow-based blending decreased the accuracy (both examples are in the direct vicinity of a captured viewpoint} \label{fig:offset_bad}
\end{figure}

\begin{figure}
\centering
    \hfill
    \begin{subfigure}[b]{\textwidth}
            \centering
            \includegraphics[width=0.9\textwidth]{04/real/stripX_5x5_random_D.jpg}
            \caption{Viewpoint ``D''}
    \end{subfigure}
    \hfill

    \hfill
    \begin{subfigure}[b]{\textwidth}
            \centering
            \includegraphics[width=0.9\textwidth]{04/real/stripX_5x5_random_M.jpg}
            \caption{Viewpoint ``M''}
    \end{subfigure}
    \hfill
  \caption{Viewpoints in the real scene where both regular and flow-based blending produced results with high error values} \label{fig:real_bad}
\end{figure}

\begin{figure}
\centering
    \hfill
    \begin{subfigure}[b]{\textwidth}
            \centering
            \includegraphics[width=0.9\textwidth]{04/real/stripX_5x5_random_I.jpg}
            \caption{Viewpoint ``I''}
    \end{subfigure}
    \hfill

    \hfill
    \begin{subfigure}[b]{\textwidth}
            \centering
            \includegraphics[width=0.9\textwidth]{04/real/stripX_5x5_random_F.jpg}
            \caption{Viewpoint ``F''}
    \end{subfigure}
    \hfill
  \caption{Viewpoints in the real scene where both regular and flow-based blending produced results with low error values} \label{fig:real_good}
\end{figure}

\begin{figure}
		\centering
    \includegraphics[width=\textwidth]{04/real/cube_5x5_random_G.jpg}
		\caption{Viewpoint ``G''}
		\label{fig:real_G}
\end{figure}

\begin{figure}
\centering
    \hfill
    \begin{subfigure}[b]{\textwidth}
            \centering
            \includegraphics[width=0.9\textwidth]{04/real/stripX_5x5_random_A.jpg}
            \caption{Viewpoint ``A'', where the flow-based blending produced worse results than the regular blending}
    \end{subfigure}
    \hfill

    \hfill
    \begin{subfigure}[b]{\textwidth}
            \centering
            \includegraphics[width=0.9\textwidth]{04/real/stripX_5x5_random_K.jpg}
            \caption{Viewpoint ``K'', where the flow-based blending produced better results than the regular blending}
    \end{subfigure}
    \hfill
  \caption{Viewpoints in the real scene where there is a clear difference between the regular and the flow-based blending results} \label{fig:real_good_bad}
\end{figure}

