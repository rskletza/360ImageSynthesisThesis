\chapter{Synthesized Images}\label{imgs}

%%%%%%%%%%%%%%%%%%%%%%% SCENARIO SCENE %%%%%%%%%%%%%%%%%%%%%%%%%%
\begin{figure}
		\centering
    \includegraphics[width=\textwidth]{04/scenario_scene/O_flow_both_scenes.jpg}
		\caption[Flow-based blending results of ``O'']{Flow-based blending result of viewpoint ``O'' in the square and oblong rooms: The bookshelf has a strong impact on the difference in error values (green) and the details of the bookshelf are warped due to inaccurate optical flow (magenta).}
		\label{fig:scene_O_flow}
\end{figure}

\begin{figure}
  \centering
  \includegraphics[width=0.9\textwidth]{04/scenario_scene/stripX_square_synth_room_N.jpg}
  \caption[Viewpoint ``N'' in the square room]{Synthesized point ``N'' in the square room (best improvement for L1, good for SSIM): The flow-based blending removes the ghosting artefacts on the rug (magenta) and improves the accuracy on the coffee table (cyan), and the lower part of the bookshelf (green). The rest of the scene is very similar for both results.}
  \label{fig:scene_square_N}
\end{figure}

\begin{figure}
  \centering
  \includegraphics[width=0.9\textwidth]{04/scenario_scene/stripX_square_synth_room_L.jpg}
  \caption[Viewpoint ``L'' in the square room]{Synthesized point ``L'' in the square room (worst ``improvement'': slight increase of error for both metrics): The ghosting artefact on the rug in the regular blending result is replaced by a displacement artefact in the flow-based blending result (magenta), which also introduces a new artefact, namely the warped top edge of the coffee table (green). Otherwise the scenes are very similar.}
  \label{fig:scene_square_L}
\end{figure}

\begin{figure}
  \centering
  \includegraphics[width=0.9\textwidth]{04/scenario_scene/stripX_oblong_room_A.jpg}
  \caption[Viewpoint ``A'' in the oblong room]{Synthesized point ``A'' in the oblong room (best improvement): The flow-based blending drastically improves ghosting artefacts on the couch (magenta) and the accuracy of the rug (cyan), but also introduces new artefacts (green).}
  \label{fig:scene_oblong_A}
\end{figure}

\begin{figure}
  \centering
  \includegraphics[width=0.9\textwidth]{04/scenario_scene/stripX_oblong_room_L.jpg}
  \caption[Viewpoint ``L'' in the oblong room]{Synthesized point ``L'' in the oblong room (worst improvement): The flow-based blending result introduces some severe discontinuity artefacts on the rug (magenta) and on the coffee table (cyan), although the coffee table is positioned more accurately in the flow-based blending result (green)}
  \label{fig:scene_oblong_L}
\end{figure}

%%%%%%%%%%%%%%%%%%%%%%% SCENARIO DENSITY %%%%%%%%%%%%%%%%%%%%%%%%%%

\begin{figure}
		\centering
    \includegraphics[width=\textwidth]{04/scenario_density/strip_dens_T_regular.jpg}
		\caption[Regular blending results for viewpoint ``T'' with different densities]{The regular blending results for point ``T'' (one of the worse results) in the square room with a viewpoint density of 2x2, 6x6, and 12x12: The 2x2 image shows the bookshelf from the wrong perspective (green), and the walls are noticeably warped (cyan), which is improved in the 6x6 and 12x12 images. The bookshelf is more accurate in the 12x12 image than the 6x6 image (magenta), but still shows some inaccuracies.}
		\label{fig:density_regular_T}
\end{figure}

\begin{figure}
		\centering
    \includegraphics[height=0.8\textheight]{04/scenario_density/strip_dens_G_regular.jpg}
		\caption[Regular blending results for viewpoint ``G'' with different densities] {The regular blending results for point ``G'' (one of the better results) in the square room with a viewpoint density of 2x2, 6x6, and 12x12: Both the accuracy and the ghosting effects are visibly reduced, the higher the density of the captured viewpoints is. This is especially clear for the coffee table (green).}
		\label{fig:density_regular_G}
\end{figure}

\begin{figure}
  \centering
  \includegraphics[width=0.9\textwidth]{04/scenario_density/stripX_2x2v2_offset_new_T.jpg}
  \caption[Viewpoint ``T'' in the 2x2 setup]{Synthesized point ``T'' in the 2x2 setup (best improvement for L1, good for SSIM): The flow-based blending synthesized a blurry approximation of the bookshelf (green), but also distorted and blurred the rest of the image (magenta), due to inaccurate optical flow from captured viewpoints with a too-large distance.}
  \label{fig:dens_2x2_T}
\end{figure}

\begin{figure}
  \centering
  \includegraphics[width=0.9\textwidth]{04/scenario_density/stripX_2x2v2_offset_new_K.jpg}
  \caption[Viewpoint ``K'' in the 2x2 setup]{Synthesized point ``K'' in the 2x2 setup (worst ``improvement'': slight increase of error): Although the regular blending result shows many doubling artefacts and incorrect positioning problems, the results of the flow-based blending are visually much worse due to inaccurate optical flow.}
  \label{fig:dens_2x2_K}
\end{figure}

\begin{figure}
  \centering
  \includegraphics[width=0.9\textwidth]{04/scenario_density/stripX_12x12_offset_new_Y.jpg}
  \caption[Viewpoint ``Y'' in the 12x12 setup]{Synthesized point ``Y'' in the 12x12 setup (best improvement for L1 and SSIM): The flow-based blending result synthesizes the bookshelf slightly more accurately (magenta), and also does not display an artefact present in the regular blending result (green). Other than that, the results are very similar.}
  \label{fig:dens_12x12_Y}
\end{figure}

\begin{figure}
  \centering
  \includegraphics[width=0.9\textwidth]{04/scenario_density/stripX_12x12_offset_new_H.jpg}
  \caption[Viewpoint ``H'' in the 12x12 setup]{Synthesized point ``H'' (worst ``improvement'': slight increase of error): The flow-based blending result synthesizes the coffee table and white pillow with cleaner edges (magenta), however, due to a bug in an external library, the flow-based blending result contains some black lines (green) that possibly skew the error values in favor of the regular blending.}
  \label{fig:dens_12x12_H}
\end{figure}

%%%%%%%%%%%%%%%%%%%%%%%%%%%%%% OFFSET SCENARIO %%%%%%%%%%%%%%%%%%%%%%%%%%%%%%%%%%

\begin{figure}
  \centering
  \includegraphics[width=0.9\textwidth]{04/scenario_offset/stripX_6x6_dense_7.jpg}
  \caption[Viewpoint 7 of 625 in the square room]{Synthesized point 7: The accuracy of the pictures on the wall is better in the flow-based blending result (magenta), and the shapes and position of the rug and the coffee table are also more accurate (green). However, the black line artefacts (cyan) are also fairly severe.}
  \label{fig:offset_7}
\end{figure}

%\begin{figure}
%  \centering
%  \includegraphics[width=0.9\textwidth]{04/scenario_offset/stripX_6x6_dense_439.jpg}
%  \caption[Viewpoint 439 of 625 in the square room]{Synthesized point 439: }
%  \label{fig:offset_439}
%\end{figure}

\begin{figure}
  \centering
  \includegraphics[width=0.9\textwidth]{04/scenario_offset/stripX_6x6_dense_283.jpg}
  \caption[Viewpoint 283 of 625 in the square room]{Synthesized point 283: Apart from a severe displacement artefact in the regular blending image, which also decreases the accuracy on the coffee table, the results are very similar.}
  \label{fig:offset_283}
\end{figure}

\begin{figure}
  \centering
  \includegraphics[width=0.9\textwidth]{04/scenario_offset/stripX_6x6_dense_145.jpg}
  \caption[Viewpoint 145 of 625 in the square room]{Synthesized point 145: The flow-based blending introduced a slight displacement on the coffee table (magenta) and a distortion on the blue table (green). Otherwise the results are very similar.}
  \label{fig:offset_145}
\end{figure}

\begin{figure}
  \centering
  \includegraphics[width=0.9\textwidth]{04/scenario_offset/stripX_6x6_dense_504.jpg}
  \caption[Viewpoint 504 of 625 in the square room]{Synthesized point 504: The position of the coffee table in the flow-based blending result is less accurate than in the regular result (green). Otherwise the results are almost identical, except for the black lines caused by the external library bug (magenta).}
  \label{fig:offset_504}
\end{figure}

%%%%%%%%%%%%%%%%%%%%%%%%%%%%%%%%%%%% REAL SCENE %%%%%%%%%%%%%%%%%%%%%%%%%%%%%%%
\begin{figure}
  \centering
  \includegraphics[height=0.9\textheight]{04/real/stripX_5x5_random_I.jpg}
  \caption[Viewpoint ``I'' in the real scene]{Viewpoint ``I'' (relatively low values for both regular and flow-based blending): Most of the scene is fairly accurate for both blending techniques. The largest positional inconsistencies are the tripod (magenta) and outside of the windows (cyan). The exernal library bug also causes very visible artefacts in the flow-based result (green).}
  \label{fig:real_I}
\end{figure}

%\begin{figure}
%  \centering
%  \includegraphics[width=0.9\textwidth]{04/real/stripX_5x5_random_F.jpg}
%  \caption[Viewpoint ``F'' in the real scene]{Viewpoint ``F'' (relatively low values for both regular and flow-based blending): }
%  \label{fig:real_I}
%\end{figure}

\begin{figure}
  \centering
  \includegraphics[width=0.9\textwidth]{04/real/stripX_5x5_random_D.jpg}
  \caption[Viewpoint ``D'' in the real scene]{Viewpoint ``D'' (high values for both regular and flow-based blending): The high error values are due to the proximity to the cabinet, which was not reprojected correctly in the regular blending, and for which the optical flow algorithms also seems to have failed (magenta).}
  \label{fig:real_D}
\end{figure}

%    \begin{subfigure}[b]{\textwidth}
%            \centering
%            \includegraphics[width=0.9\textwidth]{04/real/stripX_5x5_random_M.jpg}
%            \caption{Viewpoint ``M''}
%    \end{subfigure}

\begin{figure}
		\centering
    \includegraphics[width=\textwidth]{04/real/cube_5x5_random_G.jpg}
		\caption[Viewpoint ``G'' in the real scene]{Viewpoint ``G'' (L1 much higher for flow-based blending, SSIM only slightly higher): Most elements of the scene are more accurate in the flow-based result (magenta), however, it does introduce some artefacts, including on the door and wall (green) and on the ceiling lamp, where the optical flow was inaccurate (cyan).}
		\label{fig:real_G}
\end{figure}

\begin{figure}
  \centering
  \includegraphics[width=0.9\textwidth]{04/real/stripX_5x5_random_A.jpg}
  \caption[Viewpoint ``A'' in the real scene]{Viewpoint ``A'' (error values higher for flow-based blending): The flow-based blending result shows some ghosting artefacts due to failed optical flow (magenta), and discontinuities (green), but also has a more accurate positioning of the cabinets (cyan).}
	\label{fig:real_A}
\end{figure}

\begin{figure}
  \centering
  \includegraphics[width=0.9\textwidth]{04/real/stripX_5x5_random_K.jpg}
  \caption[Viewpoint ``K'' in the real scene]{Viewpoint ``K'' (flow-based blending produced better results than the regular blending): The shape of the lamp is more accurate in the flow-based blending result (green), and the whole area around the lamp (the small table and the edge between the walls and floor is also in a more accurate position.}
	\label{fig:real_K}
\end{figure}

