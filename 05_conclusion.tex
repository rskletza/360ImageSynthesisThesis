\chapter{Conclusion}

The approach and implementation presented in this thesis laid the groundwork for a pixel-based 2-DoF synthesis using 1-DoF interpolation. 
The goal of using the 1-DoF interpolation was to mitigate artefacts caused by the inaccuracy of the proxy geometry by generating viewpoints with more accurate perspectives.
The results of the evaluation showed that the flow-based blending was able to improve the accuracy of the results in the majority of cases where the basic reprojection produced significant artefacts.
%However, the evaluation also uncovered some of the limitations of the flow-based approach, predominantly some severe artefacts based on ...

Using the insights gained in the evaluation, it will be possible to improve the 2-DoF synthesis with flow-based interpolation, for example by combining it with cutting-edge deep learning technologies, and leveraging its potential for parallelization. will make it possible to synthesize images in real-time.
This could potentially enable casually captured environments to be experienced interactively, which could enhance a broad range of Virtual Reality applications.


%The evaluation provided insights on the strengths and limitations of this approach, which can be used in the future in order to improve the proof-of-concept implementation.
%casually captured environments could be experienced interactively, which could enhance a broad range of Virtual Reality applications.


