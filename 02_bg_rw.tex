\chapter{Background and Related Work}

\section{Terminology}

\paragraph{Planar image}
\paragraph{360\degree image}
\paragraph{Interpolation vs Synthesis}
\paragraph{Viewpoint, capture, camera} input viewpoint, output viewpoint
Capture/Viewpoint is the term used here to signify the location and image data of a 360\degree photograph taken at a certain position in the scene
capture is more for when the images are actually recorded
viewpoint is more for during processing

\section{Fundamentals of 360\degree Images}

\subsection{Projections for Planar Viewing \label{projections}}
\begin{itemize}
    \item sphere, latlong/equirectangular, cube
    \item interpolation \ar working with pixels \ar want minimal distortion
    \item sphere \& latlong have extreme distortion around edges/poles
    \item cube has some distortion in corners/edges, but a lot less than the other projections
    \item introduce world coordinates vs image coordinates
\end{itemize}

\section{Related Work}
