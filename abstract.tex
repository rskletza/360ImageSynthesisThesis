\vspace*{2cm}

\begin{center}
    \textbf{Abstract}
\end{center}

\vspace*{1cm}

\noindent 
Virtual Reality technology allows users to experience virtual environments by interacting with them, and navigating within them. These environments tend to be either meticulously modeled in 3D by hand, or prerecorded using 360\degree cameras. The advantage of using 360\degree images is that they achieve a high level of realism with little effort, however, they often limit the user to a single viewpoint.
Image-based rendering, or image-based synthesis aims to create novel viewpoints based on captured viewpoints, in the best case enabling a user to navigate freely around a scene.
There are a number of different approaches to image-based synthesis, many using some form of feature correspondence to extract information, such as scene geometry, from the captured images. Extracting scene geometry from images can be problematic, since inaccurate scene geometry can lead to severe artefacts.
%Other approaches use no geometry, which may require a very high amount of captured data, or 
This thesis proposes a pixel-based 2-DoF synthesis algorithm that combines basic reprojection using proxy geometry with flow-based interpolation.
The use of simple proxy geometry, such as a sphere, in place of estimated scene geometry makes it possible to synthesize novel views within a scene without knowledge about the geometry.
However, the difference of the proxy geometry from the scene geometry can lead to severe artefacts, just like when using inaccurately estimated scene geometry.
Flow-based interpolation, which is generally used for interpolating between pairs of images, is leveraged in order to alleviate some of these artefacts.
A proof-of-concept implementation of the approach is presented, and tested with a select set of parameters, using different virtual and real scenes.
The results are then evaluated based on mathematical error metrics, as well as visible artefacts. The results of the evaluation show that the flow-based method improves the basic method in a number of cases.

